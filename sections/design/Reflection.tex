\chapter{Reflection}
In this chapter, different aspects of the report are being reflected upon.

\section{Ideal Robot}
In this report, NASA's rovers has been researched and described. This helped the project along by giving an idea of which components was vital for Hades to be able to explore an area. The group did not have the time or the resources for creating Hades. For this reason the ideal requirements was delimited for fitting into the turtlebot.

\section{Turtlebot}
There was some technical difficulties connecting to the turtlebot. Some participators could not get connection to the turtlebot of various reasons. One problem that occurred was that the Q&A site for ROS was down for about two weeks. This resulted in the group not being able to solve the problems when being stuck in that period of time.
Due to the hardware of the turtlebot, most of the requirements from Hades had to be delimitated. The turtlebot was the basic idea of how the group wanted Hades to work, by moving, mapping and avoiding obstacles.
%The group did not have the money to have better hardware like a lidar on the turtlebot.

%As mentioned above this implementation of the software have been working well, but its another story for the hardware. Considering that it is the first semester a turtlebot should suffice, but for the delimitations, see \ref{ch:Delimitation}, there are not much to work with. If the turtlebot should have contributed, it shouldn't just run on flat surfaces and the sensors should be for outdoor use.
%May the force be with you

\section{Test}
The testings worked out as planned. We experienced some problems with test 5, read section \ref{ch:discussionTest5}. Other than that, tests succeeded as planned. 
The different communication systems was tested, such as:
\begin{itemize}
    \item \texttt{2D\_nav}
    \item \texttt{Publish Points}
    \item \texttt{Teleop}
\end{itemize}


\section{The V-model}
As the work progressed in the group, it was decided to use the "V-model" for chapter \ref{ch:design}.
It divides the three sections, section \ref{ch:designDetection}, \ref{ch:designPathplanning} and \ref{ch:designMovement}, into blocks. These blocks have sub blocks, which can clarify the main block.\\ 
This helped the group to get a better understanding of the whole chapter and made it easier to push forward.