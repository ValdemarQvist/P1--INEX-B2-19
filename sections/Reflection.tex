\chapter{Reflection}

\section{Ideal Robot}

NASA's rovers has been described and this has helped the project getting into the right state of mind. This project has been working on how to give the same "feel" to the turtlebot. It can never reach anything like NASA's rovers, but the software and implementation of this in the tests, has given a better understanding of how it could work.\\

\section{Turtlebot}

As mentioned above this implementation of the software have been working well, but its another story for the hardware. Considering that it is the first semester a turtlebot should suffice, but for the delimitations, see \ref{ch:Delimitation}, there are not much to work with. If the turtlebot should have contributed, it shouldn't just run on flat surfaces and the sensors should be for outdoor use.\\ 

\section{Test}

The testing section worked as planned, we experienced a couple of failures, but mostly the robot worked as planned.\\
The group tested different velocities and the effect it had on the map. The different communication systems was tested as well, such as:\\
2D\_nav\\
Publish Points\\
and\\
Teleop.\\
To conclude, the tests showed that the theory matched the acceptance tests.\\

\section{The V-model}

As the work progressed in the group, we decided to use the "V-model" for structuring our Design. This model was given to the group by supervisor Jesper Abildgaard.\\
It divides the three main chapters in to blocks which has sub blocks, which can clarify the main block.\\
This helped the group get a better understanding of the whole design and made it easier to push forward.\\


