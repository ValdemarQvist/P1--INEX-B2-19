\chapter{Environments on Mars}\label{ch:environmentOnMars}

Mars, also called the Red Planet, was named by the Romans after the god of war because of its red, blood-like color. The martian terrain and its climate will be described in this chapter.

\section{Terrain} 
The martian terrain contains notable elements which will be explained in this section.

\subsection{Hellas Basin, and other craters}
Impact craters are remnants of past collisions with asteroids on the surface of mars. The craters vary in size, the largest is named Hellas basin, which is 2253 km in diameter and its floor is about 7152 m  deep and the height to the top from the bottom of the rim is 9000 metres.The surroundings is a circular rim structure and looks like mountains with spaces between them.\\
Hellas basin is thought to have formed about 4.1 to 3.8 billion years ago, during a period when large asteroids were hitting the surface of Mars \cite{hellas}.

\subsection{Chaotic terrain}\label{ch:chaoticT}
On Mars there exist chaotic terrain, which is unlike anything we have on earth.\\
Chaotic terrain is terrain that has been created by old volcano flows having undermined the ground where it flowed, and then again at a later time ice will have covered the old flows. But new eruptions has then created new undermined flows of lava which ran through the old ones, and hereby created this special type of chaotic terrain\cite{CTerrain}.%layer upon layer
%many flows. 

\subsection{Ice caps}
Seasonal ice caps consist of carbon monoxide, water and dust, which covers both of Mars poles and reaches up to heights of 3 km and are estimated to be as wide as 1000 km.

\subsection{Possible caves}\label{ch:caves}
Observing from odyssey spacecraft, NASA scientists have identified seven possible caves.\\
The entrance of these pits are 100 to 252 meters wide and they are believed to be around 70 to 96 meters deep\cite{surface}\cite{guide}.

\subsection{Soil and water}\label{soil}
Mars surface is covered with rocks, fine sandy silt and dust similar to some regions on earth.\\
The chemical composition of Mars soil can differ depending on where a sample has been taken, but the most known element found is iron in the form of rusting iron oxide, which gives the planet its famous red color.\\
All the evidence is suggesting that there used to be water on Mars in the past, and that it has been found in all of its three states: solid ice, gas and occasionally liquid in some specific areas\cite{liquid}.\\
More than five million km${^3}$ of water in the form of ice has been identified close to the surface of Mars\cite{water}.

\section{Climate of Mars}
The main composition of the martian atmosphere is carbon dioxide.\\
The surface of Mars has a low thermal inertia, meaning that it heats up fast when under the sun light. At noon and near the equator, the temperature reaches as high as 20$^{\circ}$C and at the poles it is around -153$^{\circ}$C.\\
Thermal inertia changes in some areas on Mars away from the poles, resulting in daily temperature swings and winds, which in return can pick dust particles up into the air and create clouds.\\
Mars' atmosphere has a scale height of approximately 11 km\cite{climate}.

\section{Conclusion}
To explore the Hellas Basin crater on Mars, the martian terrain has to be taken into consideration to prevent unnecessary damages to the robot, which could otherwise have been prevented.
One of the problems could be what has been described in section \ref{ch:chaoticT} where some places could be difficult for a robot to drive.\\
The robot needs to be able to avoid ice caps on the rim, since there is no guarantee for the robot to be clean from bacteria, which is explained in chapter \ref{ch:Spacelaw}. In addition to this, the robot has to be able to detect all the hazardous environments and respond appropriately \cite{AspectsWeather}.