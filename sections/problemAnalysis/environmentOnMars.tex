\chapter{Environment on Mars}\label{ch:environmentOnMars}

Mars, also called the Red Planet, was named by the Romans after the god of war because of its red, blood-like color. The martian terrain and its climate will be described in this chapter.

\section{Terrain} 
The martian terrain contains notable elements which will be explained in this section.

\subsection{Hellas Basin, and other craters}
Impact craters are remnants of past collisions with certain types of asteroid on the surface of mars. The craters vary in size, the largest is named Hellas basin which is 2253 kilometer in diameter and its floor is about 7,152 m  deep. The surrounding is a circular rim structure and look more like mountains with spaces between them.
the rim is 9,000 m higher than the bottom of the basin and the pressure at the bottom is 12.4  mbar which is  103 percent higher than the normal surface of mars  and because its above the triple point of water, liquid water could form in the basin under certain conditions. Hellas basin is thought to have formed about 4.1 to 3.8 billion years ago, during the period that large asteroids would hit Mars \cite{hellas}.

\subsection{Chaotic terrain}\label{ch:chaoticT}
On Mars there exist chaotic terrain, which is unlike anything we have on earth.\\
Chaotic terrain is terrain that has been created by old volcano flows having undermined the ground where it flowed, and then again at a later time ice will have covered the old flow but a new eruption has then created a new flow of lava which has run over the old flows, and thus created this strange terrain\cite{CTerrain}.

\subsection{Ice caps}
Seasonal Ice caps consist of carbon monoxide, water and dust, which covers both of Mars poles and can reach as high as 3 km in height and estimated to be as wide as 1000 km.

\subsection{Possible caves}
Observing from odyssey spacecraft, NASA scientists have identified seven possible caves. The entrance of these pits are 100 to 252 meters wide and they are believed to be around 70 to 96 meter in depth\cite{surface}\cite{guide}.

\subsection{Soil and water}
Mars surface is covered with rocks, sand and dust similar to some regions on earth.
The chemical composition of Mars soil can differ depending on where a sample has been taken, but the most known element found is iron in the form of rusting iron oxide, which gives the planet its famous red color.
\newline All evidence suggests that Mars used to have a watery past like that of earth, and water has been found in three states: solid ice, gas and occasionally liquid in some specific areas\cite{liquid}.
more than five million cubic kilometer of water in the form of ice  has been identified close to the surface of Mars\cite{water}.

\section{Climate of Mars}
The main composition of the martian atmosphere is carbon dioxide, and it has the surface pressure of 600 Pa where on earth, the atmospheric pressure on the surface is 101000 Pa.
\newline The surface of mars has a low thermal inertia, meaning that it heats up fast when under the sun light. At noon and near the equator, temperature reaches as high as 20$^{\circ}$C and at the poles it is around −153$^{\circ}$C. Thermal inertia changes in some areas on mars away from the poles, resulting in daily temperature swings and winds, which in return can pick up dust particles into the air and create clouds. Mars' atmosphere has a scale height of approximately 11 km\cite{climate}.
\begin{table} [h]
    \centering
    \begin{tabu} to 1\textwidth { | X[c] X[c] X[c] | }
     \hline
      & Earth & Mars \\ 
     \hline
     Length of Year & 365.25 days & 687 Earth Days \\
     \hline
     Length of Day & 23 hours 56 minutes & 24 hours 37 minutes \\
     \hline
     Gravity & 9.82m/s$^2$ & 3.711m/s$^2$ \\
     \hline
     Temperature ranges from & -89$^{\circ}$C +55$^{\circ}$C &-153$^{\circ}$C at the poles +20$^{\circ}$C at equator \\
     \hline
     Atmosphere & Nitrogen, oxygen, & Mostly carbon dioxide, \\
      & argon, others & some water vapor \\
     %\hline
     %Number of moons & 1 & 2 \\
     \hline
    \end{tabu}
    \caption{Mars compared to Earth\cite{MarsVSEarth}}
    \label{tab:marsEarthFacts}
\end{table}

\newpage

\section{Conclusion}
To explore the Hellas Basin crater on Mars, the martian terrain has to be taken into consideration to prevent unnecessary damages to the robot, which could otherwise have been prevented.
One of the problems could be what has been described in section \ref{ch:chaoticT} where some places could be difficult for a robot to drive over.\\
Another problem is that if the robot explores the northwestern rim of the crater, it could stumble upon the ice caps. The robot which has to avoid it, because there is no guarantee the robot to be clean for bacteria from earth, which is explained in chapter \ref{ch:Spacelaw}. In addition to this, The robot has to be able to detect all the definitions and respond appropriately \cite{AspectsWeather}.


%As mentioned earlier, Mars has four seasons due to the tilt of its rotational axis just like Earth. Mars’ orbit is slightly elliptical which affects the length of its seasons and is about 1.5 times farther from the sun than Earth’s orbit therefore the seasons on Mars lasts about twice as long, see figure \ref{fig:marsSeasons} for a graphic illustration of the difference between Mars and Earth when it comes to seasons\cite{MarsBasicFacts}\cite{MarsOverview}\cite{gravity}\cite{MarsInDepth}.


%As seen on figure \ref{fig:marsSeasons}, the months on Mars are not only longer than the ones on Earth, but the seasons’ positions are different as well. When there is summer and winter on Earth, Mars experiences autumn and spring. This has something to do with how the planets tilt of its rotational axis to do.\todo{rewrite.. A bit messy} Both Earth and Mars have close to the same axial tilt, but the two planets do not point in the same direction in space. This is the reason to why Mars is about one season ahead compared to Earth\cite{MarsSeasons}
.
%Make this table more beautiful later on: https://da.sharelatex.com/learn/Tables



%The average temperature of Mars is 77$^{\circ}$ Celsius colder than Earth, with an average temperature of - 63$^{\circ}$ Celsius, see table \ref{tab:marsEarthFacts}.

%In this figure \ref{fig:image003.jpg}, Earth and Mars are compared to each other as to the differences between the days in a year, gravity, sunlight and atmosphere. 

%\begin{figure}[h]
    %\centering
    %\includegraphics[width=.8\linewidth]{figures/im%age003.jpg}
    %\caption{Earth and Mars comparison %\cite{gravity}}
 %   \label{fig:image003.jpg} 
%\end{figure}



%https://mars.nasa.gov/allaboutmars/facts/#?c=inspace&s=distance

%Write about Radiation, atmosphere and sunlight
%It is important to know: atmosphere and how much sunlight reaches Mars surface






%https://www.nature.com/articles/ngeo2412





