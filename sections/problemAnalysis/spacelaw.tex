\chapter{Space Law} \label{ch:Spacelaw}
Space Law is the description of the laws governing all space-related activities.
Space Law comprises a variety of international agreements, treaties, conventions, United Nations resolutions as well as rules and regulations of international organizations. The main body of the Space Law is commonly named the “Five United Nations treaties on outer space”, which have all been signed by the three depository governments: the Russian Federation, the United Kingdom and the United States of America.

\section{The “Outer Space Treaty”}
This treaty is the main source which other treaties derive from. It covers the principles regarding exploration and use of outer space, which includes the Moon, and other celestial bodies.\\
Below is stated six articles related to driving a robot on Mars:

\subsection{The “Rescue Agreement”}
This treaty is focusing on the agreements about rescuing Astronauts, as well as returning objects launched into outer space.\\
The agreement elaborates on Article 8[appendix Articles] of the “Outer Space Treaty”, which provides that states shall take all possible steps to rescue and assist astronauts in distress, and upon requests provide assistance in recovering space objects which has returned to earth outside the territory of the launching state\cite{Treaty2}.

\subsection{The “Liability Convention”}
This treaty is an elaboration to article 7 of the “Outer Space Treaty” which states:
“Each State Party to the Treaty that launches or procures the launching of an object into outer space, including the moon and other celestial bodies, and each State Party from whose territory or facility an object is launched, is internationally liable for damage to another State Party to the Treaty or to its natural or juridical persons by such object or its component parts on the Earth, in air or in outer space, including the moon and other celestial bodies\cite{Treaty3}.”

\subsection{The “Registration Convention”}
This treaty is built upon the desire for a mechanism which can assist in identifying space objects, which has then since 1962 been upheld by the United Nations, to keep an open registry of any space object. 
To this date 92 percent of all satellites, probes, landers, manned spacecraft and space station flight elements launched into earth orbit or beyond have been registered with the Secretary-General of the United Nations\cite{Treaty4}. 

\subsection{The “Moon Agreement”} \label{ch:moonAgreement}
This Agreement reaffirms and elaborates on many of the provisions of the Outer Space Treaty as applied to the Moon and other celestial bodies, providing that those bodies should be used exclusively for peaceful purposes, that their environments should not be disrupted, that the United Nations should be informed of the location and purpose of any station established on those bodies. In addition, the Agreement provides that the Moon and its natural resources are the common heritage of mankind and that an international regime should be established to govern the exploitation of such resources when such exploitation is about to become feasible\cite{Treaty5}.

\section{Conclusion}
There are many laws considering Mars and outer space. The ones stated in this chapter are important to describe when sending out an object to outer space and also for a robot to land and drive on Mars. This chapter states that Mars must not be contaminated and the environment should not be disrupted. This  is the core reasons why the robot should avoid areas with water.\\
The "Liability Convention" is important, since it helps clarifying who is to be liable in case our robot breaks other governments equipment, or contaminate an area with micro bacteria.
