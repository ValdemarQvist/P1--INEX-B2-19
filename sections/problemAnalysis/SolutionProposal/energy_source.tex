\section{Energy source}

In space exploration there are two sources used for generating power to the space vehicles. One is solar power and the other one is a warm decaying isotope. This part will describe those two types of energy sources.

\subsection{Solar power}

Solar power is produced by solar cells. A solar cell is made of two layers of silicon atoms, one with a positive charge and the other is of negative charge. A silicon crystal holds the atoms in a grid shape. When the photons hits the surface of the cell, the photon kicks one electron from the grid on the negative charged side. Then it moves through a wire to the positive charged side, this is where the power is created. When electricity is produced it has to be used or stored in a battery, so it can be saved and used when there is more need for power. For example when there is no sun available. One down side to solar cells is that they need the sun, another is that the most efficient solar cell only converts 46\% of the light that hits the cell \cite{SolarPanels}.

\subsection{Radioisotope thermoelectric generator}

This method relies on a heating source, which comes from a decaying isotope of plutonium called Pu-238. For converting heat to electricity it uses a method called thermocouple. Thermocouple works by having two different metals attached to each other at one end and heating it, this will produce a small current.
A radioisotope thermometric generator has an efficiency around 6 to 7\% the rest is still heat and can be used to keep the core of the robot and it's tools warm. The radioisotope thermometric generator method has been used in many space missions and is a very reliable source of energy\cite{RTG}.

%\newpage
