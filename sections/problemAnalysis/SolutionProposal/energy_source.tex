\section{Energy source}

In space exploration, different types of energy sources exists to generate power for thespace vehicles. One of them is solar power and another one is a decaying isotope.\\
This part will describe two of such types of energy sources.

\subsection{Solar power}
Solar power is produced by solar cells. A solar cell is made of layers of silicon stacked on each other, one layer with positive charged atoms and the other layers is of negative charged atoms.  When the photons hits the surface of the cell, the photon kicks one electron from the layer onto the negative charged side. Then it moves through a wire to the positive charged side, which is how the power is generated. When electricity is produced it has to be used or stored in a battery, so it can be saved and used when the power is needed, for example when there are no sun available.

\subsection{Radioisotope thermoelectric generator}\label{Rtg}
This method relies on a heating source, which comes from a decaying isotope of plutonium called Pu-238. For converting heat to electricity it uses a method called thermo coupling. Thermo coupling works by having two different metals attached to each other at one end, heating up that end will produce a small current.
A radioisotope thermometric generator can be used to keep the core of the robot and its tools warm. The radioisotope thermometric generator method has been used in many space missions and is a very reliable source of energy, but also a scarce material\cite{RTG}.


