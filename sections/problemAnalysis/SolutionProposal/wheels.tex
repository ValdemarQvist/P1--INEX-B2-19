\section{Wheel systems}
The robot will need to be able to move around and therefore wheels or continuous tracks are needed. This part will describe the advantages and disadvantages of the two different types of wheel systems.


\subsection{Wheels}\label{ch:Wheels}
%\textbf{Advantages}\\
Wheels have several advantages in comparison to track systems. One of the most relevant advantages is the weight of the wheels, since they can be made in many types of materials, without losing to much durability.
Another advantage is the low production costs, and simplicity of wheel types, since they do not have too many complex moving parts. This makes them cheaper to manufacture. Because there are few moving parts, there is a chance that sand and dust from the surface of Mars will get stuck in places that will decrease the mobility. 
Being able to control the wheels individually will increase the maneuverability.

%\textbf{Disadvantages:}\\
Wheel types have a few disadvantages. One of these is when the robot has to climb over rocky terrains, it would not be able to get across vertical obstacles with at least half the height as the diameter of the wheels. On slippery surfaces, wheels will have a harder time to get a strong grip and the robot can end up getting stuck in place, or having to consume more energy to move\cite{Wheels1}\cite{Wheels2}.
%could be
\subsection{Continuous Tracks}
%\textbf{Advantages}\\
The advantages of track systems, is that they have a high power delivery efficiency, which is optimized to a high performance when going through rough terrains.
Using tracks also allows for heavy loads, because the weight is getting spread to the entire surface of the tracks, compared to wheels, the weight is only being distributed to where the wheels make contact with the ground.

%\textbf{Disadvantages}\\
On the other hand, tracks have a more complex mechanical system compared to wheel systems, which means they can break down more easily compared with wheels, if sand stone or dust gets stuck somewhere, which requires a higher degree of maintenance if it should not break down.
Another issue with track systems is that they are less precise in maneuverability, since their movements are all based on 2 points of contact to the ground. This contact spans over a larger area than wheels and for this reason it also required more power to make turning movements with track systems\cite{Wheels1}\cite{Wheels2}.

