\section{Materials}
Material used in space missions, must be light enough, and have the ability to withstand the harsh environment of the space.
Our ability to manipulate different parts of the rover and repair them if needed during the mission is highly limited due to different reasons, such as the communication delay and the complexity of a comprehensive repair mechanism. This results in rely on the use of materials that are sometimes difficult to contain or attain. A good example to this would be the radioisotope of certain elements used in thermo-electric batteries which are extremely complicated to handle yet more reliable then solar panels\cite{NuclearPower}. Due to the radiation level inside the robot we need to pick electrical components, that is not effected by the radiation or in some cases there is a need for shielding these components.
There is a different mentality for material used in space in a sense that, we are not using the same kind of material if we were to build a rover to explore caves or inaccessible areas on earth. To help understand this we could consider that, erosion could have different meanings on mars compare to earth.


\section{Conclusion}
In overall, to avoid obstacles, a robot needs a camera as eyes. The LIDAR sensor is one of the most sufficient method for scanning the sorroundings.\\
As for the moving part, either wheels or continuous tracks can be used as a solution. Both types have advantages and disadvantages. Wheels are faster, more precise in maneuverability but are more complex in its construction than continuous track.\\
The energy source for a robot is important. Solar power is dependent on sunlight, which means sun is needed in this process for generating power. The RTG method can work in any environment, meaning that it does not depend on sunlight. It is mentioned earlier than Pu-238 is a reliable source of energy, but the efficiency with RTG method is much lower than the efficiency of solar panels. Both methods has advantages and disadvantages.\\
