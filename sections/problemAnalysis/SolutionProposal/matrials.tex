\section{Materials}
Materials used in space missions must have the ability to withstand the harsh environment of space.\\
The ability to manipulate different parts of the rover and repair them if needed during the mission is highly limited due to different reasons, such as the communication delay and the complexity of a comprehensive repair mechanism. This results in the reliability of materials that sometimes are difficult to contain or attain. A good example to this would be the radioisotope of certain elements used in thermo-electric batteries which are extremely complicated to handle yet more reliable than solar panels\cite{NuclearPower}. \\ Due to the radiation level inside the robot coated electrical components is needed, so that it is not effected by the radiations.\\
Different kind of material would be used to explore caves or inaccessible areas on earth. Considering that, erosion could have different meanings on mars compared to earth.


\section{Conclusion}
To avoid obstacles, a robot needs a camera to perceive the surroundings. The LIDAR sensor, with a 360$^{\circ}$ field of vision and looking up to 120 m into the distance, is a useful feature for scanning the surroundings and detecting obstacles.\\
As for the moving parts, either wheels or continuous tracks can be used as a solution. Both types have advantages and disadvantages. Wheels are faster, more precise in maneuverability but are more complex in its construction than continuous tracks.\\
The energy source solar power is dependent on sunlight, which means sun is needed in this process for generating power. The RTG method can work in any environment, meaning that it does not depend on sunlight. It is mentioned earlier than Pu-238 is a reliable source of energy, where solar panels wear out after years.

