\chapter{Existing solutions}\label{ch:existingSolutions}
There has already been multiple rovers sent to Mars. A set of solutions which already exists that can explore Mars will be introduced in this chapter.

\section{Sojourner}\label{ch:existingSolutions_SojournerRover}
Sojourner was a six-wheeled rover that was the first to roam Mars.
Sojourner was built with a non-rechargeable battery that had a capacity of 150 Wh and solar cells with the size 0.22 m${^2}$, which could deliver around 15 W on the surface of Mars.\\ As communication Sojourner used an ultra high frequency radio modem, which communicated with the lander, Pathfinder. Sojourner had three cameras on board with different lenses, two in front, which were monochrome and a color camera in the rear. Also an alpha proton x-ray spectrometer was implemented, it could tell the composition of the chemical elements present on the surface of Mars, but it could not detect hydrogen. Sojourner weighed 11.5 kg and was 65 cm in length. It had a mission length of 7 days, but it surpassed the planned mission duration and kept on working until day 85. Sojourner traveled more than 100 meters during its 85 day long mission. It took three years to build, and had a cost of 175 million dollars\cite{Sojounerroverjpl}.

\section{Spirit and Opportunity} \label{ch:existingSolutions_SpiritOpportunity} 
The second generation of rovers that landed on mars were the two twin rovers Spirit and Opportunity.\\
The mission of the two rovers was meant to last for at least 90 days, Spirit’s mission ended in 2014, but Opportunity's mission continued.\\ 
Opportunity has covered over 45 km on the surface of Mars. The rovers were meant to run 40 m in a day, or 1 cm per second. The wheels in the front and in the back could turn independently of each other, so the rover had a better turning ratio.\\
The rovers had two radioactive hardened CPU's, one main, and one for backup. Furthermore, a 256 MB flash memory for storing pictures and data before it was sent back to Earth\cite{spiritopportunity_overview}.\\
They had nine cameras and six different scientific tools at disposal. One of the instruments was the panoramic camera that used two cameras with different color filters to detect structure and minerals. Another was the navigation tool, it used two monochrome cameras for navigation and driving, due to a wider field of view but with a lower resolution than the color cameras.\\
The rovers kept themselves warm with a radioisotope heater that could generate 1 W of thermal energy.\\
For communication, the rovers had three different antennas, two for communicating with earth, and one for communication with the orbiter. The power was harvested from solar panels, which could recharge the batteries. These panels got covered by small dust particles but the weather on Mars is creating small dust devils, which help the rovers to become dust-free again\cite{Marsdust}.

\section{Curiosity}
\label{ch:existingSolutions_Curiosity}
Curiosity is the current rover on Mars. This vehicle was sent to Mars November 2011 and landed on Mars August 2012. Curiosity's sole purpose is to investigate the climate and geology of the Gale Crater\cite{CuriosityMissions}.\\ Curiosity is semi-autonomous and is intelligent enough to detect water and obstacles. One downside is that the satellites orbiting Mars only have the capacity to communicate with the rover sixteen hours a day due to the lack of orbiters communicating with Earth. When the scientists at NASA communicates with the rover it will take fourteen minutes for the signal to reach the rover\cite{CuriosityCommunication} \cite{CuriosityNASA}.\\
The rover found a water stream in 2016. NASA calculated that the rover itself might contaminate the stream, and since they couldn't sterilize Curiosity with a 100\% certainty, they were scared that Earth bacteria would travel through the air and contaminate the entire water environment\cite{Curiosity2016}.

\subsection{Instruments}
The thermal system of the rover is quite complex. The key components are covered in electrical heaters, and the heat and power is created by the isotope of plutonium-238. Curiosity can drill into boulders and also vaporize samples to examine it's structure. When Curiosity drills, it will move the sample from the drill to either SAM or CheMin, which are built-in laboratories for the rover to handle the samples taken.\\
In these bullet-points beneath you can read about the specific cameras NavCam and HazCam \cite{CuriosityNASA}\cite{CuriosityPOWER}: 

\begin{itemize}
\item Four Hazcams is mounted on the lower part of Curiosity. The HazCams use visible light to collect the data of the surroundings, they will then create a three-dimensional imagery of this data.\\ These cameras are created to help fine tune the movements for Curiosity in close proximity with hazardous obstacles, with a 120${^\circ}$ vision for each of the four cameras. The 3D imagery builds a map of the terrain with three meters and four as the farthest. It is mapping in triangular shapes, and placing the cameras in each direction this will make a square map for the artificial intelligence to work in\cite{CuriosityVision}. 
\item The two pairs of NavCams is mounted on the mast of Curiosity. As the HazCams they use visible light to structure a 3D map. They have a 45${^\circ}$ field of vision, and the reason they are implemented is to correlate with the HazCams, and help structure the images taken of the terrain \cite{CuriosityVision}. 
\end{itemize}

\section{Conclusion}
As presented the rovers of Mars has given insight into different components which could be vital for a successful mission, like the different cameras and sensors to navigate through the terrain, the different types of power sources aswell as the type of wheels used..\\ 
What worked well in these previous robots and what is going to be implemented in the ideal robot for this project, will be elaborated in chapter \ref{ch:solutionProposal}.