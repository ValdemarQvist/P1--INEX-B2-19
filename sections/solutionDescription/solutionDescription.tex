\chapter{Solution Description}\label{ch:solutionDescription}

 

This chapter will introduce and describe the different elements chosen for the final design. Furthermore it will be explained why these parts have been chosen compared to the other possible components.

\section{Final problem formulation}\label{ch:finalproblem}
%How is it possible to design an autonomous vehicle that can move around in an unknown area. While the vehicle has to autonomously navigate and avoid obstacles.
%How can a robot be designed so that it can map, move in an unknown area and avoid obstacles? 
%How is it possible to design a rover that can move in an unknown area while map and avoid obstacles?
%How is it possible to design a rover that uses SLAM for moving in an unknown area while mapping and avoiding obstacles?
%It is possible to create a robot with existing solutions, who? can avoid obtacles and map the area by his own in unknown places?

How is it possible to design a rover that uses SLAM for automatically moving in an unknown area while mapping and avoiding obstacles, but also possible to be controlled manually.



\section{Design delimitation}\label{ch:Designdelimitations}

Below is stated delimitations for the solution proposal:

\begin{itemize}
    \item The robot has to operate in an unknown outdoor area.
    \item The robot has to stay clear of obstacles with a height of more than 5 cm. 
    \item The robot has to be able to climb a grading of 7\% without getting stuck.
    \item If an obstacle is not avoidable the robots has to call for human interaction.
    \item The robot may not move more than 5 meters pr 14 minutes due to time delay.
\end{itemize}

\section{Design Requirement Specifications} \label{ch:Designrequiremnts}

The robot is required to have enough space to fit scientific tools inside the base.\\  
The distance between ground and chassis should be at least 50 cm and the width of the robot should be the same as an average car.\\ 
The robot needs a field of vision of at least 10 meters, since the hazards of sink holes and cliffs should be present.\\
The maximum velocity of the robot should be 0,006 m/s so the operators have a chance to stop the robot interfering with hazardous areas.\\
The camera will need a 360 degree of view implemented under the base. A way to apprehend damages to the camera by the environment, the camera will be surrounded with acrylic panzer glass \todo{source for self-repairing glass}\cite{Lidar360}.\\ 
The robots internal has to be protected from radiation, since radiation can be more or less damaging to different materials\cite{radiationEffectsInMaterials}.\\
The robot needs a redundant circuit of electronics, such as if the system breaks down a backup system can still operate the robot.\\
The wheels of the robot needs to have a high friction due to the sand surface on Mars\cite{sand}.

\section{Design proposal}\label{ch:Designproposal}
This section will give a short description of how the robot should function.\\
The robot needs motors to accelerate, decelerate and overcoming the incline that is set at a maximum of 7\%. For vision the robot will use a LIDAR sensor for detecting and ranging the terrain.\\
For the robot to move through the perceived environment, it will need a algorithm to assist moving autonomously.\\
