\chapter{Implementation theory}

\section{Software}

\subsection{Rviz} \label{ch:Frontier Exploration}

\subsection{Mapping}

\subsection{Localization}

\subsection{SLAM}

SLAM is not a specific algorithm, but a concept where there are many different approaches. SLAM uses a collection of data gathered from varies sensors such as odometry and distance measuring sensors, to build a map and to know where it is located in the map.
There are different solutions to SLAM some are made for indoor use others for outdoor, underwater, and some is made to navigate in an area with terrain. Odometry is a measurement of how much the wheels have rotated. This can be unreliable due to wheels are slipping or other reasons. Furthermore, this "noise" in the signal can be dealt with using it in combination of odometry and a gyroscope that measures the rotation and movement in different directions.\\
When the robot starts mapping it looks with its sensors, such as cameras, LIDAR, laser, etc. for objects. After, the robot moves forward, measures how far it moved with the odometry sensor and gyroscope, then looks at on the surrounding area to see new relation with objects\cite{SLAMdummies}\cite{DifferentSLAM}\cite{GyrosOdometry}.\\
A simple algorithm for SLAM could be:
\begin{itemize}
    \item Scan area for objects
    \item Move forward and measure how far
    \item Scan again for objects
    \item Look for reappearing objects
    \item Triangulate location to objects that reappeared
    \item Relocate the robot in the map 
\end{itemize}

\subsection{Algorithm for avoid obstacle}


