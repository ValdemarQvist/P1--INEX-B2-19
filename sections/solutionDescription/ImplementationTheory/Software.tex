\section{Software}
The robot will need different components of software to be able to move, map the area it is moving in, and avoid obstacles. This section will introduce the different types of software that is needed for doing as mentioned above.

\subsection{ROS}
ROS is used to interact with robots. This operation system comes with a collection of libraries and different commands to execute the communication.\\
The idea with ROS is to connect topics to each other through nodes. On each node there is a topic that you can subscribe or publish to. When there is published or subscribed on a topic it is received in a message that contain the data of this topic. The data can then be used in different programs, to interact with the robot or get information about location and visualization. \\
Before all of this can work there is the need for ROS-master. The ROS-master keeps track of every topic that has been subscribed or published to \cite{ROSwiki}.

\subsection{Rviz}
Rviz is a program used by ROS to provide 3D images and point clouds from data given by the robots sensors. Rviz lets the user spectate the robot in different ways. The input the robot gets from cameras and sensors is numbers and coordinates. Rviz converts these numbers into 3D objects which represent these objects\cite{interactiveMarkers}.

%\subsection{Mapping}
%ROS has a mapping package called gmapping. The gmapping package provides laser-based SLAM, as a ROS node called slam\_gmapping. Furthermore, slam\_gmapping creates a 2D map in Rviz with lasers and position data collected by the robot\cite{mapping}.

%\subsection{Localization}
%Robot localization is a task to move safely from one location to another.
%It helps determine the Turtlebot's position by using AMCL. The localization system AMCL  is collecting information from the sensors such as odometry.\\By comparing data from sensors to the known map, it is possible to estimate the current position of the robot.\\ The robot visualize a map to find the trajectory to move forward. The two processes are commonly referred to as SLAM. The localization system, AMCL, working-process is described in figure \ref{fig:amcl}.


% During operation, AMCL estimates the transformation of the base frame in respect to the global frame. But it only publishes the transformation between the global frame and the odometry frame\cite{AMCL}.

\subsection{SLAM}

SLAM is not a specific algorithm, but a concept where there are many different approaches. That is why there is different solutions for SLAM. Some are made for indoor use, some for outdoor use.\\
SLAM uses a mapping package \textit{gmapping} in ROS. The gmapping package in ROS is called slam\_gmapping. This package uses data collected from lasers and the robots position to create a 2D map. %Furthermore, slam\_gmapping creates a 2D map in Rviz with lasers and position data collected by the robot.\\
When the robot starts mapping, the localization system AMCL helps the robot to move safely from one location to another.
Information from sensors, such as odometry and gyroscope, is collected when mapping. 

By comparing data from sensors to the known map, it measures how far it has moved. Moreover, looks at the surrounding area to check its new relative position.\\The robot visualize a map to find the trajectory to move forward. The localization system, AMCL, working-process is described in figure\ref{fig:amcl}.

\begin{figure}[h]
    \centering
    \includegraphics[width=.7\textwidth]{figures/AMCL.png}
    \caption{AMCL working process\cite{AMCL}} 
    \label{fig:amcl} 
\end{figure}

 During operation, AMCL estimates the transformation of the base frame in respect to the global frame. But it only publishes the transformation between the global frame and the odometry frame\cite{AMCL}.



% it will then look with its sensors, LIDAR, laser, etc. for objects. After the robot moves forward, it measures how far it has moved with the odometry sensor and gyroscope, then looks at the surrounding area to check its new relative position with objects\cite{SLAMdummies}\cite{DifferentSLAM}\cite{GyrosOdometry}.\\
%A simple algorithm for SLAM could be:
%\begin{itemize}
%    \item Scan area for objects
%    \item Move forward and measure how far
%    \item Scan again for objects
%    \item Look for reappearing objects
%    \item Triangulate location to objects that reappeared
%    \item Relocate the robot in the map 
%\end{itemize}

%use a collection of data gathered from various measuring sensors, to build a map and to triangulate itself inside this map.
%SLAM is not a specific algorithm, but a concept where there are many different approaches. SLAM use a collection of data gathered from various measuring sensors, to build a map and to triangulate itself inside this map.

\subsection{Frontier Exploration}

FrontierExploration package will be used as a algorithm for our robot to navigate through a perceived environment.\\
The Package provides a costmap2D, client/server nodes and Layer BoundedExplorerLayer. The BoundedExplorerLayer can be used for many complex explorations, functioning by two services. Updatepolygonfrontier, this is where the input is telling the robot which area to scan. Then the service GetNextFrontier, will wait for user input and then run the first service again.\\
The robot in this project has no global map, which means the robot should be able to create the map on its own. This can be done autonomously by publishing a starting point, it will then determine its own path and act accordingly. Furthermore, it will still be able to allow user input, such as a published polygons \cite{ROSexploration}.
\newpage

