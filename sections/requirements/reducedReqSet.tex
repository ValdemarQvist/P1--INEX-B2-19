\chapter{Reduced Requirement set} \label{ch:Reduced Requirement Set}

\section{Turtlebot} %introduction
A Turtlebot is a prototype robot made for developers. In this project the Turtlebot will be used to test the different implementations which the project needs, due to the immense task of building a real rover in time for this project.\\
The Turtlebot consist of different parts, which will be described in this section.

\subsection{Base} %chassis
\begin{figure}[h]
    \centering
    \includegraphics[width=.5\textwidth]{figures/turtlebot001.png}
    \caption{The turtlebot with new setup} 
    \label{fig:turtlebot} 
\end{figure}
The base is a Kobuki that is made for prototyping and educational use. The base has two motors, which can move the robot at a velocity of 0.65 m/s.The motor is 12V brushed DC motor that is controlled by a H-bridge that is connected to the source. The motors have an encoder on each wheel, which makes 11.5 ticks per millimetre of movement and 2578.33 ticks per full rotation. The motors are protected from high current above 3 amp. The maximum rotational velocity is 180 deg/s. The base has a gyro which works on the Z axis (110 deg/s calibrated from factory).\\
The base has three cliff sensors these sensors are located in the center and to left and right. The base has a wheel drop sensor that detect if a wheel has dropped into a hole. The Base wheels can drive over obstacles with a height of 12mm and the base can clear obstacle with a height of 15 mm. The base has two different batteries both on 14.8 VDC one has a capacity of 2200 mAh and the second one has a capacity of 4400 mAh.\\ 
The load capacity of the turtlebot is 5 kg on hard surface and on soft surfaces the load capacity is 4kg\cite{Base}.



\subsection{Computer} 
The computer is an ASUS notebook, it has an Intel core i3-4010U CPU that has 2 cores and operate at 1.7MHz\cite{CPU}. The notebook has 4GB RAM and an integrated HD graphics card on the motherboard\cite{ASUS}.
The operating system is Ubuntu 16.04 LTS a layer for the Linux distribution. For communication between the sensors and the Turtlebot, the computer uses USB 2.0.
\begin{figure}[h]
   \centering
    \includegraphics[width=.6\textwidth]{figures/ASUS.jpg}
    \caption{The computer that comes with the turtlebot}
    \label{fig:ASUS}
\end{figure}

\section{Requirements}

\begin{itemize}
    \item Have communication between host and user computer. 
    \item Have manual control.
    \item Detecting obstacles.
    \item Automatically make a path on user requests.
    \item Autonomously navigate and unknown environment.
    \item It has to be able to carry a payload.
\end{itemize}