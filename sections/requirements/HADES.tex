\chapter{Design of Hades}
Hades is the ideal robot. This is the robot the group wants to send up on Mars.\\
Here is a short introduction to the different parameters, that should be implemented in Hades.

\section{Detection}
In chapter \ref{ch:detectionHades}, the detection of Hades will be stated.\\
Here the previous mentioned sensors and software for detection will be elected towards the most efficient way of detecting obstacles.

\section{Movement}
The ideal hardware design for Hades will be stated in chapter \ref{ch:MovementHades}.\\
This will give a conclusion to the hardware design and why the group chose the different components.

\section{Conclusion}
To conclude, the different components will be reviewed and implemented in the chapters below.\\
To conclude properly the group will then refer to the requirements for Hades.

\section{Design Requirement Specifications} \label{ch:Designrequiremnts}
\begin{itemize}
    %\item The robot is required to have enough space to fit scientific tools inside the base. 
    %\item The distance between ground and chassis should be at least 50 cm and the width of the robot should be the same as an average car. 
    \item The robot needs the vision of the distance of at least 10 meters, since the hazards of sink holes, cliffs and other hazardous areas could be present.
    \item A camera with 360 degree of view should be implemented under the base \cite{Lidar360}. %A way to apprehend damages to the camera by the environment, the camera will be surrounded with acrylic panzer glass\cite{Lidar360}.
    %\item The robots internal has to be protected from radiation, since radiation can be more or less damaging to different materials\cite{radiationEffectsInMaterials}.
    %\item The robot needs a redundant circuit of electronics, such as if the system breaks down a backup system can still operate the robot.
    \item The wheels of the robot needs to have a high friction due to the sand surface on Mars\cite{sand}.
\end{itemize}

\chapter{Detection} \label{ch:detectionHades}

In this chapter the implementation of sensors for detection will be overlooked and referred to the requirements.

\section{LIDAR}
To get a accurate 3-D image of a distance-view of above 10 meters, see \ref{ch:Designrequiremnts}, Hades will need a LIDAR sensor.\\
LIDAR offers a wide variety of hardware to implement. Many of these cameras has a high spinning rate and offers a precise cloud of 3-D data.\\
These precise readings will help the robot to measure accurate distances towards objects. This will give the robot more maneuverability and make the remote control more manageable when the pictures received are clear. This data cloud will also help the robot to move autonomously.\\
A good example is the Velodyne HDL-64, which can collect 1.3 million 3-D points per second, and has a rotation rate of 10.4 HZ. With a 360 degree field of vision and looking up to 120 meter into the distance, this would be a very solid hardware implementation for Hades. \cite{Lidar360}\\
This is why the LIDAR sensor has been chosen.

\section{Radar}
Radars use concentrated radio waves, these waves can be transmitted with a high or low frequency.\\ 
To avoid obstacles the radio wave is typically at a frequency of a I or J band. This means that the high frequency will measure the electromagnetic energy emitted with higher accuracy.\\
This is what has been chosen since it has such highly concentrated waves, see chapter \ref{ch:solutionProposal}.\\
To collect a 360 field of view, the radars will be set up in a manner that the degree of vision would be equal to 360 see \ref{ch:Designrequiremnts}.


\section{Conclusion}
Solutions to the requirement of distance and 360 degrees of cameras and sensors, will be fulfilled by either one of the options.\\
The LIDAR and Radar has been chosen to cover and fulfill the detection requirements for Hades.

\chapter{Movement} \label{ch:MovementHades}

In this chapter the different movement components will be stated.

\section{Odometry}
There is a visual odometry and a wheel based odometry. The aim in this project is to combine the feed from both the visual and wheel based odometry to generate a better localization of the robot in the map.

\section{Wheels}
For the HADES project, wheels will be used as the means of to deliver traction for the robot, due to the maneuverability, endurance and low weight, See \ref{ch:Wheels}.\\
The wheels will be made of a hollow mesh that is formed in a wheel Sharp couture, so the wheels is not prone to puncture. Furthermore the wheels needs a design that makes the robot able to move in sand and be able to drive over rocks, see \ref{ch:Designrequiremnts}. \\
As stated the wheels is equipped with odometry and should therefor be able to detect itself in the perceived map. Hereby, with help from the sensors, Hades should be able to stop before entering hazardous environments.\\

\section{RTG}
RTG offers a dual solution of generating electricity for Hades, but also as a heating component to protect the robots sensory equipment from freezing up.

\section{Conclusion}

With respect to \ref{ch:Designrequiremnts}, the wheels will have a high friction due to the hollow mesh wheels.\\
Furthermore the problem with self-triangulating towards hazardous environments, the odometry will be chosen so that this will fulfill the requirements.\\
