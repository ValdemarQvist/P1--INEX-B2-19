\chapter{Discussion}

\section{Test 1}
The turtlebot had difficulties in being controlled when the speed was higher than 0.4 m/s. By driving with a slower speed, the turtlebot would be easier to control.
The group experienced to be more careful with the controls when driving at a high speed, since the turtlebot had a delay time and did not react immediately.

\section{Test 2}
When the turtlebot was moving faster, the Rviz had difficulties keeping up with the robots movements and also with making the map. Rviz is not working as fast as it would need to, when the turtlebot moves with a speed of 0.8 m/s or faster. The turtlebot had complications driving over the carpet with a speed of 0.2 m/s but none with a speed of 0.4 m/s and faster.

\section{Test 3}
The test showed that, with the parameter Max\_vel\_x 0.2 the turtlebot made a detailed map and could avoid obstacles ease.\\
One test showed that, the parameter Max\_vel\_x can be set to low.

\section{Test 4}
Test 4 showed that the robot with the parameter set as 0.2 would make a faster path-plan and a more detailed map. It would also avoid obstacles that was difficult to detect for the robot, such as a staircase with no connection to the ground.\\
This is the most efficient settings that have been tested on the turtlebot so far.